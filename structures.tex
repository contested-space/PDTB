\documentclass{article}
\usepackage[francais]{babel}
\usepackage[utf8]{inputenc}
\usepackage[T1]{fontenc}
\usepackage{tipa}
\usepackage{amsmath, amssymb}
\usepackage{hyperref}
\usepackage{listings}

\setlength\parindent{0in}
\setlength\topmargin{-0.25in}
\setlength\headheight{0in}
\setlength\headsep{0in}
\setlength\topskip{0.5in}
\setlength\textheight{9.75in}
\setlength\textwidth{6.50in}
\setlength\oddsidemargin{0in}
\setlength\evensidemargin{0in}

\begin{document}
\section{Rythme changeant}
L'idée ici est de trouver une formule pour calculer la longueur de la structure
suivante :

$L_a seq(1) L_a seq(2) ... L_a seq(n) L_a seq(n-1) ... L_a seq(2)$

Où $L_a$ est une structure melodique statique de a notes et $seq(k)$ est une 
structure melodique de n notes dont seulement les k premières sont jouées.

L'intéret de cette structure est de générer une forme mélodique qui contient
des éléments toujours reconnaissables qui s'éloignent et se rapprochent
continuellement. 

On sait que pour une séquence complète de taille n, $L_a$ sera joué $2(n-1)$ fois.
Le nombre total de notes généré par les itérations de $L_a$ sera donc $2a(n-1)$.

La quantité de notes générée par seq est définie par 
$\sum\limits_{i=1}^n i + \sum\limits_{i=2}^{n-1} i = n + 2\sum\limits_{i=1}^{n-1} i - 1$.
Comme l'identitée $\sum\limits_{i=1}^n i = \frac{n(n+1)}{2}$ est connue, nous
obtenons la quantité $2a(n-1) + n + 2\sum\limits_{i=1}^{n-1} i - 1= n^2 +2an - 2a - 1$.

Maintenant que cette quantité est connue, il  nous est possible (en supposant 
que toutes les notes ont une durée égales) de trouver ses diviseurs, de manière
à y superposer une ou plusieurs voix tout en gardant une cohérence rythmique.

Il nous est maintenant possible de tester la divisibilité de la structure pour
un certain entier b en posant $n^2 + 2an - 2a - 1 = bk$ (b et k entiers). 

Par exemple, si on souhaite utiliser cette structure avec $n=5$ et $a=3$, la
quantité totale de notes est $(5)^2 + 2(3)(5) - 2(3) - 1 = 48$, ce qui permet
d'y superposer des boucles de 2, 3, 4, 6, 12, 16 et 24 temps. 

On pourrait aussi espérer trouver une façon de superposer cette structure à 
elle-même, avec des paramètres différents. Il nous suffit de trouver
$a_0, a_1, n_0, n_1$ tels que 
$k_0(n_0^2 + 2 a_0 n_0 - 2 a_0 - 1) = k_1(n_1^2 + 2 a_1 n_1 - 2 a_1 - 1)$. À
première vue, une équation quadratique au second degré avec 6 variables peut 
sembler peu pratique. Il est en fait possible que cette équation soit un
problème de la classe \texttt{NP-hard}. Cependant, comme les résultats qui
nous intéressent sont dans les entiers relativement bas (par exemple, pour
accomplir une superposition de boucles d'une minute, il nous faut seulement les
résultats dont le total est de 60), il nous est possible d'en obtenir les 
combinaisons possibles par force brute. Il est trivial d'écrire un programme
qui génère toutes les combinaisons possibles jusqu'à un certain point.
Par exemple:

\begin{lstlisting}
for(k0 = 1; k0 <= ; k0++)
  for(k1 = 1; k1<= Nk1; k1++)
    for(a0 = 1; a0<=Na0; a0++)
      for(a1 = 1; a1<=Na1; a1++)
      	for(n0 = 1; n0<=Nn0; n0++)
      	  for(n1 = 1; n1<=Nn1; n1++)
       	  	if(k0*(n0*n0 + 2*n0*a0 - 2*a0 - 1) 
       	  		== k1*(n1*n1 + 2*n1*a1 - 2*a1 - 1) 
       	        && !(a0 == a1 && n0 == n1) && !(n0 == 1 && n1 == 1))
          	  printf("k0 = %d, k1 = %d, a0 = %d, a1 = %d, n0 = %d, 
          	  	      n1 = %d, total0 = %d, total1 = %d \n",
          	         k0, k1, a0, a1, n0, n1, 
          	         n0*n0 + 2*n0*a0 - 2*a0 - 1, 
          	         n1*n1 + 2*n1*a1 - 2*a1 - 1);

\end{lstlisting}
\subsection{Quelques combinaisons possibles}
\begin{lstlisting}
k0 = 1, k1 = 1, a0 = 1, a1 = 4, n0 = 9, n1 = 7, total0 = 96, total1 = 96
k0 = 1, k1 = 1, a0 = 1, a1 = 5, n0 = 6, n1 = 4, total0 = 45, total1 = 45
k0 = 1, k1 = 1, a0 = 1, a1 = 6, n0 = 5, n1 = 3, total0 = 32, total1 = 32
k0 = 1, k1 = 1, a0 = 1, a1 = 9, n0 = 4, n1 = 2, total0 = 21, total1 = 21
k0 = 1, k1 = 1, a0 = 1, a1 = 9, n0 = 9, n1 = 5, total0 = 96, total1 = 96
k0 = 1, k1 = 1, a0 = 2, a1 = 6, n0 = 7, n1 = 5, total0 = 72, total1 = 72
k0 = 1, k1 = 1, a0 = 2, a1 = 8, n0 = 5, n1 = 3, total0 = 40, total1 = 40
k0 = 1, k1 = 1, a0 = 2, a1 = 10, n0 = 10, n1 = 6, total0 = 135, total1 = 135
k0 = 1, k1 = 1, a0 = 3, a1 = 7, n0 = 8, n1 = 6, total0 = 105, total1 = 105
k0 = 1, k1 = 1, a0 = 3, a1 = 10, n0 = 5, n1 = 3, total0 = 48, total1 = 48
k0 = 1, k1 = 1, a0 = 4, a1 = 1, n0 = 7, n1 = 9, total0 = 96, total1 = 96
k0 = 1, k1 = 1, a0 = 4, a1 = 8, n0 = 9, n1 = 7, total0 = 144, total1 = 144
k0 = 1, k1 = 1, a0 = 4, a1 = 9, n0 = 7, n1 = 5, total0 = 96, total1 = 96
k0 = 1, k1 = 1, a0 = 4, a1 = 10, n0 = 6, n1 = 4, total0 = 75, total1 = 75
k0 = 1, k1 = 1, a0 = 5, a1 = 1, n0 = 4, n1 = 6, total0 = 45, total1 = 45
k0 = 1, k1 = 1, a0 = 5, a1 = 9, n0 = 10, n1 = 8, total0 = 189, total1 = 189
k0 = 1, k1 = 1, a0 = 6, a1 = 1, n0 = 3, n1 = 5, total0 = 32, total1 = 32
k0 = 1, k1 = 1, a0 = 6, a1 = 2, n0 = 5, n1 = 7, total0 = 72, total1 = 72
k0 = 1, k1 = 1, a0 = 7, a1 = 3, n0 = 6, n1 = 8, total0 = 105, total1 = 105
k0 = 1, k1 = 1, a0 = 8, a1 = 2, n0 = 3, n1 = 5, total0 = 40, total1 = 40
k0 = 1, k1 = 1, a0 = 8, a1 = 4, n0 = 7, n1 = 9, total0 = 144, total1 = 144
k0 = 1, k1 = 1, a0 = 9, a1 = 1, n0 = 2, n1 = 4, total0 = 21, total1 = 21
k0 = 1, k1 = 1, a0 = 9, a1 = 1, n0 = 5, n1 = 9, total0 = 96, total1 = 96
k0 = 1, k1 = 1, a0 = 9, a1 = 4, n0 = 5, n1 = 7, total0 = 96, total1 = 96
k0 = 1, k1 = 1, a0 = 9, a1 = 5, n0 = 8, n1 = 10, total0 = 189, total1 = 189
k0 = 1, k1 = 1, a0 = 10, a1 = 2, n0 = 6, n1 = 10, total0 = 135, total1 = 135
k0 = 1, k1 = 1, a0 = 10, a1 = 3, n0 = 3, n1 = 5, total0 = 48, total1 = 48
k0 = 1, k1 = 1, a0 = 10, a1 = 4, n0 = 4, n1 = 6, total0 = 75, total1 = 75
\end{lstlisting} 






\end{document}